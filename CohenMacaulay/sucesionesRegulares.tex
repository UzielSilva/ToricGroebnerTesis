\section{Sucesiones regulares y complejos de Koszul}

Una definición muy importante para poder trabajar, es la de \emph{sucesión regular}, ya que extiende la noción de un elemento que no es divisor de cero. A lo largo de esta sección a estos elementos les llamaremos \emph{regulares}

\begin{definition}

Sea $R$ anillo y $M$ $R$-módulo. Una sucesión de elementos $x_1,\dots,x_n \in R$ es una \emph{sucesión regular} en $M$ (o una $M$-sucesión) sí:

\begin{enumerate}[i)]
\item $(x_1,\dots,x_n)M \neq M$
\item $(\forall i \in \{1,\dots, n\})(x_i \textrm{ es regular en } M/(x_1,\dots, x_{i-1}))$
\end{enumerate}

\end{definition}

Estudiaremos esta noción con una herramienta homológica llamada \emph{complejo de Koszul}. Su definición es algo complicada, por lo que primero trataremos de entender los casos más simples para después generalizar.

\subsection{Definiciones}

Antes de continuar, necesitaremos definiciones tanto para entender la definición particular del complejo de Koszul de longitud 1 y 2, como para entender la definición general(no constructiva) para cualquier longitud. Para evitar la repetición innecesaria de definiciones, asumiremos que $R$ es un anillo.

%\begin{definition}
%Dado $R$ anillo, un \emph{$R$-módulo} $M$ es un grupo abeliano con una acción de $R$, es decir, una función $R \times M \rightarrow M$, denotada como $(r,m) \rightarrow rm$, que satisface las siguientes propiedades:

%\[  \]
%\end{definition}
%\begin{definition}

\begin{definition}
Sean $M, N$ submódulos de un $R$-módulo $P$, definimos al ideal cociente como

$$(M:N) := \{r \in R|rN\subset M\}.$$

Abusaremos de la notación y escribiremos $(x:y)$ en lugar de $((x):(y))$. Si $I \subset R$ es un ideal y $M \subset P$ es un submódulo, usaremos la notación $(M:I)$ ó $(M:_PI)$ para referirnos al submódulo $\{p \in P|pI \subset M\}$. 

\end{definition}

\begin{definition}
Sea $M$ un $R$-módulo, el \emph{anulador} de $M$, denotado como $\ann(M)$, se define como

$$ \ann(M) = (0:M) = \{r \in R | rM = 0\}. $$

\end{definition}


\begin{definition}

Un \emph{complejo de cocadenas de $R$-módulos} (o simplemente \emph{complejo}) $C^{\bullet}$ es una sucesión de módulos $\{C^i\}_{i \in \mathbb{Z}}$ y funciones $\{d^i: C^{i-1}\rightarrow C^i\}_{i \in \mathbb{Z}}$ tal que, para toda $i \in \mathbb{Z}$, $d^{i+1} \circ d^i = 0$. En ocasiones a estas funciones se les denomina como diferenciales.

Nos referimos al kernel de $d^n$ como $Z^n(C^\bullet)$, y a su imagen como $B^n(C^\bullet)$.

La \emph{cohomología} $H^i$ de este complejo en $C^i$ es el módulo:

$$ \ker(d^{i+1})/\textrm{im}(d^{i})$$

\end{definition}

Un complejo, en efecto, puede usar índices finitos, si lo completamos con módulos cero y morfismos cero.

También, cuando estudiamos relaciones entre complejos es inevitable encontrarnos en algún momento con diagramas conmutativos, lo que nos lleva a las siguientes definiciones.

\begin{definition}

Un \emph{morfismo de cadenas} entre dos complejos $F^{\bullet}$ y $G^{\bullet}$ es una sucesión de homomorfismos de módulos $\{f^{\bullet} = {f^i:F^i \rightarrow G^i}\}_{i \in \mathbb{Z}}$ tal que, para cada $i \in \mathbb{Z}$, el siguiente diagrama conmuta:

\[
\begin{tikzcd}
F^{i-1} \arrow[d,"d^i"] \arrow[r,"f^{i-1}"] & G^{i-1} \arrow[d,"d^i"]\\
F^i \arrow[r,"f^i"] & G^i
\end{tikzcd}
\]

Utilizamos el mismo símbolo $d^i$ para referirnos a los diferenciales correspondientes de cada complejo.

\end{definition}

\begin{definition}

Una \emph{sucesión exacta de complejos}

$$ \cdots \rightarrow C^{\bullet}_{i-1} \xrightarrow{f^{\bullet}_i} C^{\bullet}_i \xrightarrow{f^{\bullet}_{i+1}} C^{\bullet}_{i+1} \rightarrow \cdots$$

Es una sucesión de morfismos de cadenas $\{f^{\bullet}_i:C^{\bullet}_{i-1}\rightarrow C^{\bullet}_i\}_{i \in \mathbb{Z}}$ tales que, para cada $j\in \mathbb{Z}$, la sucesión de homomorfismos de módulos

$$ \cdots \rightarrow C^{j}_{i-1} \xrightarrow{f^j_i} C^{j}_i \xrightarrow{f^j_{i+1}} C^{j}_{i+1} \rightarrow \cdots$$

es exacta, es decir, para toda $i \in Z$, $\ker(f^j_{i+1}) = \im(f^j_i)$. También se le denomina \emph{sucesión exacta corta} a una sucesión exacta que consta de sólamente dos morfismos $f^\bullet:C\rightarrow D$, $g^\bullet:D\rightarrow E$ consecutivos distintos de cero, representándose de la siguente manera:

$$ 0 \rightarrow C^{\bullet} \xrightarrow{f^{\bullet}} D^{\bullet} \xrightarrow{g^{\bullet}} E^{\bullet} \rightarrow 0$$

\end{definition}

Para hablar de la definición general del complejo de Koszul necesitaremos unas cuantas definiciones de álgebra multilineal.

\begin{definition}
Sean $M, N$ $R$-módulos, definimos al producto tensorial $M \otimes_{R} N$ como el $R$-módulo generado por el conjunto 

$$\{m \otimes n | m \in M, n \in N\}$$

bajo la relación:

$$(am + a'm')\otimes(bn + b'n') = ab(m\otimes n) + a'b(m'\otimes n) + ab'(m\otimes n') + a'b'(m'\otimes n')$$

en donde 

$$ r(m \otimes n) = (rm) \otimes n = m \otimes (rn) $$

Cuando el anillo es inferido por el contexto, podemos denotarlo simplemente como $M \otimes N$.
\end{definition}

\begin{definition}
Sea $R$ anillo, $N$ un $R$-módulo, y $k \in \mathbb{Z}^*$; definimos la $k$-ésima potencia tensorial como el producto tensorial de $N$ consigo mismo $k$ veces:

$$ T^kN = N \otimes N \otimes \cdots \otimes N $$

Por convención decimos que $T^0N = R$. El álgebra tensorial de $N$ es la suma directa de $T^kN$ para $k = 0, 1, 2, \dots$

$$ T(N) = \bigoplus_{k=0}^{\infty} T^kN $$
\end{definition}

\begin{definition}
Sea $N$ un $R$-módulo, el álgebra exterior $\wedge N$ de $N$ se define como el álgebra tensorial $T(N)$ módulo las relaciones $x \otimes y = -y \otimes x$ y $ x \otimes x = 0$ para toda $x, y \in N$. El producto de dos elementos $a, b \in \wedge N$ se denotará como $a \wedge b$.
\end{definition}

$\wedge N$ es un álgebra graduada, las componentes de grado $k$, que corresponden a $T^k(N)$ bajo la relación de equivalencia, se denotan como $\wedge^kN$. Además, es anticonmutativa en el sentido de que si $a$ y $b$ son elementos homogéneos, entonces

$$a \wedge b = (-1)^{(\deg a)(\deg b)} b\wedge a$$

y si $a$ es de grado $1$, entonces $a \wedge a = 0$. Abusaremos de la notación y escribiremos $(-1)^{ab}$ en lugar de $(-1)^{(\deg a)(\deg b)}$.

La construcción de $\wedge N$ es funtorial. Esto significa que si tomamos $f: N \rightarrow M$ un morfismo de módulos, podemos definir al morfismo de álgebras $\wedge f: \wedge N \rightarrow \wedge M$ mandando $a \wedge b \wedge \dots$ a $fa \wedge fb \wedge \dots$.

Por último, si $N$ es un módulo libre de rango $n$, entonces $\wedge^n N \cong R$, y si $f: N \rightarrow N$ es un morfismo, entonces $\wedge^nf$ es la multiplicación por el determinante de alguna matríz de representación de $f$. Además, $\wedge^mN = 0$ para $m > n$.

\subsection{Complejos de Koszul de longitud 1 y 2}

Podemos decidir si un elemento $x \in R$ es regular a través de la homología del complejo

$$ K(x):0\rightarrow R\xrightarrow{x} R $$

la cual resulta ser $(0:x)$. Esta sencilla observación es la base para nuestro estudio de las sucesiones regulares a través de la homología.

Dado un segundo elemento $y \in R$, la multiplicación por $y$ define una función entre complejos $K(x)\rightarrow K(x)$, es decir, un diagrama conmutativo

\[
\begin{tikzcd}
K(x):0 \arrow[r,""] & R \arrow[r,"x"] \arrow[d,"y"] & R \arrow[d,"y"]\\
K(x):0 \arrow[r,""] & R \arrow[r,"x"] & R
\end{tikzcd}
\]

Podemos usar la conmutatividad del cuadrado en el diagrama para construir un complejo más grande, el cual lo representamos esquemáticamente de la siguiente manera

\[
K(x,y):
\begin{tikzcd}
0\arrow[r,""] & R \arrow[r,"x"] \arrow[rd,"y"] & R \arrow[d,draw=none,"\bigoplus" description] \arrow[r,""] \arrow[rd,"y"] & 0\\
&0 \arrow[r,""] & R \arrow[r,"-x"] & R \arrow[r] & 0
\end{tikzcd} \label{complex2schema} \tag{$\star$}
\]

o en una notación más usual como 

\[
K(x,y):
\begin{tikzcd}[ampersand replacement=\&]
0 \ar[r] \& R \ar{r}{
\left( \begin{smallmatrix} y \\ x \end{smallmatrix} \right)
} \& R \oplus R \ar{r}{
\left( \begin{smallmatrix} -x & y \end{smallmatrix} \right)
} \& R
\end{tikzcd} \label{complex2common} \tag{$\star\star$}
\]

En la literatura se suele definir de otra forma al complejo de Koszul, usualmente se representa con los signos de la siguiente manera

\[
K(x,y):
\begin{tikzcd}[ampersand replacement=\&]
0 \ar[r] \& R \ar{r}{
\left( \begin{smallmatrix} y \\ -x \end{smallmatrix} \right)
} \& R \oplus R \ar{r}{
\left( \begin{smallmatrix} x & y \end{smallmatrix} \right)
} \& R
\end{tikzcd}
\]

Estas dos posibilidades de representarlo, sin embargo, son isomorfas como complejos, por lo que no representa un problema. Nosotros elegiremos la primera forma por conveniencia para cálculos que analizaremos más adelante.

Podemos ver de la definición que $H^0(K(x)) = (0:x)$, el anulador de $x$, y $H^0(K(x,y))$ es $(0, (x,y))$, así que si $x$ es regular entonces $H^0(K(x,y)) = 0$.

Ahora analizemos qué es $H^1(K(x,y))$. Primero analicemos el kernel de la función de la derecha. Un elemento $(a,b) \in R\oplus R$ está en el kernel sí y sólo si $-xa+yb = 0$, por lo que $b \in (x:y)$. De igual manera, si $b \in (x:y)$, entonces hay un elemento $a$ tal que $-xa + yb = 0$, así que $(a,b)$ estará en el kernel. Si asumimos que $x$ es regular, entonces $a$ está únicamente determinada por $b$, y la asociación $b\rightarrow a$ es un homomorfismo de módulos, así que el kernel es isomorfo a $(x:y)$.
Por otro lado, un elemento está en la imagen de la función de la derecha sí y sólo si es de la forma $(cy, cx)$, así que los elementos de $(x:y)$ que corresponden a los elementos de la imagen son los elementos de $(x)$. Así, si $x$ es regular, entonces

$$H^1(K(x,y))\cong(x:y)/(x)$$

En particular, si $x$ es regular entonces $H^1(K(x,y)) = 0$ sí y sólo si la sucesión $x, y$ satisface la condición (II) en la definición de sucesión regular.

Aún podemos extraer más información si dirigimos nuestra atención al diagrama (\ref{complex2schema}): La fila de abajo es un subcomplejo(es decir, sus diferenciales no se salen de él) isomorfo a $K(x)$, mientras que la fila de arriba, también isomorfa a $K(x)$, es el cociente de $K(x, y)$ por la fila de abajo. En efecto, el $i-1$-ésimo termino del subcomplejo $K(x)$ está incluído en el $i$-ésimo térmido de $K(x, y)$, el cual proyecta hacia el $i$-ésimo término del cociente $K(x)$, por lo que tenemos la siguiente sucesión exacta corta:

$$ 0 \rightarrow K_1(x) \xrightarrow{i^\bullet} K(x,y) \xrightarrow{p^\bullet} K_2(x) \rightarrow 0 $$

Además, un famoso lema llamado el \emph{"lema de la serpiente"} establece que es posible obtener una sucesión exacta larga de homologías a partir de una sucesión exacta corta de complejos, redefiniendo a los morfismos $f^i:C^i\rightarrow D^i$ como $H(f^i):H^i(C^{\bullet})\rightarrow H^i(D^{\bullet}), H(f^i)(x + B^i(C^\bullet)) = f^i(x) + B^i(D^\bullet)$ (el lema también garantiza que estos nuevos morfismos están bien definidos), así que esta sucesión exacta larga obtenida tiene la forma:

$$\cdots \rightarrow H^0(K_2(x)) \xrightarrow{\delta} H^0(K_1(x)) \rightarrow H^1(K(x, y)) \rightarrow H^1(K_2(x)) \rightarrow \cdots$$

donde la función $\delta$ es el "homomorfismo que conecta". Para explicar este homomorfismo nos apoyaremos en el siguiente diagrama, cuyas columnas representan a $K_1(x), K(x, y)$ y $K_2(x)$, respectivamente:

\[
\begin{tikzcd}[ampersand replacement=\&]
\&\& 0 \ar{d} \& 0 \ar{d} \\
\&0 \ar{r} \ar{d} \& R \ar{r}{p^0} \ar{d}{
\left( \begin{smallmatrix} y \\ x \end{smallmatrix} \right)
} \& R \ar{d}{x} \ar{r} \& 0 \\
0 \ar{r} \&R \ar{r}{i^1} \ar{d}{-x} \& R \oplus R \ar{r}{p^1} \ar{d}{
\left( \begin{smallmatrix} -x & y \end{smallmatrix} \right)
} \& R \ar{d} \ar{r} \& 0 \\
0 \ar{r} \&R \ar{r}{i^2} \ar{d} \& R \ar{r}{p^2} \ar{d} \& R \ar{r} \ar{d} \& 0 \\
\& 0 \& 0 \& 0 
\end{tikzcd}
\]

Donde $i^1(z) = \left(\begin{smallmatrix} z \\ 0 \end{smallmatrix} \right)$, $p_1$ es la proyección de la segunda coordenada, y $p^0, i^2, p^2$ son $id_{R}$. Ahora, sea $z \in H^0(K_2(x))$. Para obtener su imagen a través de $\delta$, debemos tomar su preimagen a través de $H(p^0)$, que es el mismo $z$, y luego aplicar a un representante de la clase el diferencial. En este caso la imagen es trivial y sólo tenemos un representante, así que obtendremos $\left(\begin{smallmatrix} yz \\ xz \end{smallmatrix}\right)$. Pero recordemos que $z \in Z^0(K_2(x))$, así que $z \in (0:x)$ y $\left(\begin{smallmatrix} yz \\ xz \end{smallmatrix}\right)=\left(\begin{smallmatrix} yz \\ 0 \end{smallmatrix}\right)$, que pertenece al kernel de $p^1$, así que, como $H(i^1)$ y  $H(p^1)$ forman una sucesión exacta, tomemos la preimagen de $\left(\begin{smallmatrix} yz \\ 0 \end{smallmatrix}\right) + B^1(K(x,y))$ a través de $H(i^0)$, que es $yz$, siendo finalmente la imagen de $\delta$, por lo tanto, $\delta(z) = yz$.

Ahora supongamos solamente que $H^1(K(x, y)) = 0$. Se sigue de la sucesión exacta de homologías que:

$$H^0(K(x))/yH^0(K(x)) = 0$$

En general, esto no nos dice mucho, pero si suponemos además que $R$ es un anillo noetheriano local y $y$ está en el ideal maximal, entonces, por el lema de Nakayama(falta hablar de esto) tenemos que $H^0(K(x)) = 0$. Por ende, $x$ es regular, y $x, y$ forman una sucesión regular por lo visto anteriormente. Enunciaremos lo discutido anteriormente en el siguiente teorema.

\begin{theorem}\label{regular-homology}
Si $R$ es un anillo noetheriano local, y $x, y$ están en el ideal maximal, entonces $x, y$ forman una sucesión regular sí y sólo si $H^1(K(x, y)) = 0$.
\end{theorem}

Por la manera en la que definimos al complejo de Koszul en (\ref{complex2common}), es claro que los complejos $K(x, y)$ y $K(y, x)$ son isomorfos. Así que, bajo la hipótesis del teorema anterior, $x, y$ es una sucesión regular sí y sólo si $y, x$ lo es. Esto es suficiente para mostrar que las sucesiones regulares permiten permutaciones.

\begin{corollary}\label{m-sequence-permutation-corollary}
Si $R$ es un anillo noetheriano local, y $x_1, \dots, x_r$ es una sucesión regular de elementos en el ideal maximal de $R$, entonces cualquier permutación de $x_1, \dots, x_r$ es sucesión regular.
\end{corollary}

\begin{proof}
Como toda permutación es producto de transposiciones de elementos vecinos, basta con probar que el corolario se cumple cuando transponemos dos elementos vecinos; es decir, si $x_1,\dots, x_i, x_{i+1}, \dots, x_r$ es una sucesión regular, entonces $x_1,\dots, x_{i+1}, x_i, \dots, x_r$ también lo es. Para esto basta con probar que $x_{i+1}, x_i$ es sucesión regular módulo $(x_1,\dots, x_{i-1})$, pero gracias a la propiedad (II) de la definición de sucesión regular, $x_i, x_{i+1}$ es una sucesión regular módulo $(x_1,\dots, x_{i-1})$, y gracias al teorema \ref{regular-homology} la sucesión permutada es regular.
\end{proof}

A primera impresión podemos pensar que la hipótesis de anillo local en estos dos resultados está de más, pero no es el caso.

\begin{example}\label{counterexample-nolocal}
Consideremos el anillo

$$R = K[x,y,z]/(x-1)z$$

y la sucesión de elementos 

$$x, (x-1)y.$$

El ideal que generan es $(x, (x-1)y) = (x,y) \neq R$. Más aún, es fácil de ver que $x$ es regular en $R$, y $R/(x) = K[x,y]/(z)$. Así, $x, (x-1)y$ forman una sucesión regular, y

$$H^1(K(x, (x-1)y)) = 0$$

Sin embargo, $(x-1)y$ es un divisor de cero(se anula al multiplicarse con $z$), así que la sucesión en orden inverso no es regular.
\end{example}
Algo importante de destacar es lo siguiente: Si $x \in R$ es arbitrario, entonces $H^0(K(x, 0)) = H^0(K(x))$(ya que ambos son isomorfos a $(0:(x))$) incluso si el complejo $K(x, 0)$ no es isomorfo a $K(x)$.

\subsection{Complejos de Koszul en general}

Podemos construír al complejo de Koszul paso a paso, iterando el proceso antes mencionado, pero la siguiente construcción es más directa, simple, e invariante, por lo que tiene más ventajas. Después veremos que tanto esta deficinión como la definición iterativa son equivalentes.

\begin{definition}

Dado un módulo $N$ y un elemento $x \in N$, definimos al \emph{Complejo de Kozsul} como el complejo

$$ K(x):0 \rightarrow R \rightarrow N \rightarrow \wedge^2N \rightarrow \cdots \rightarrow \wedge^iN \xrightarrow{d_x} \wedge^{i+1}N \rightarrow \cdots $$

donde $d_x$ manda un elemento $a$ al elemento $x\wedge a$; en particular, $1 \in R$ es mandado a $d_x(1) = x \in N$. Si $N$ es libre de rango $n$ y

$$ x = (x_1, \dots, x_n) \in R^n \cong N$$

entonces en ocasiones escribiremos $K(x_1,\dots,x_n)$ en lugar de $K(x)$.

\end{definition}

Una de las ventajas de esta definición es que hace obvias las propiedades funtoriales del complejo de Koszul. En efecto, si $f:N \rightarrow M$ es un morfismo de módulos que manda $x \in N$ a $y \in M$, entonces el morfismo $\wedge f: \wedge N \rightarrow \wedge M $ preserva diferenciales, e induce un morfismo de complejos, por ser un morfismo de álgebras.

Para familiarizarnos más con el complejo de Koszul, y dado que será importante después, mostremos que $H^n(K(x_1,\dots,x_n)) = R/(x_1,\dots,x_n)$. Sea $N = R^n$, y consideremos el lado derecho del complejo de Koszul:

$$ \cdots \rightarrow \wedge^{n-1}N \rightarrow \wedge^nN \rightarrow \wedge^{n+1}N = 0 $$

Sea $e_1, \dots, e_n$ una base para $N = R^n$. Tenemos que $\wedge^nN \cong R$ por el isomorfismo que manda $e_1\wedge\cdots\wedge e_n$ a $1$, además que $Z^n(K(x_1,\dots,x_n)) =  \wedge^nN$, por lo tanto basta con probar que $B^n(K(x_1,\dots,x_n))$ es isomorfo a $(x_1,\dots,x_n)$. De igual manera, $\wedge^{n-1}N\cong R^n$, con base $e_1\wedge\cdots\wedge e_{i-1}\wedge \hat{e}_i\wedge e_{i+1}\wedge\cdots\wedge e_n$, para $i = 1,\dots,n$, en donde el símbolo $\hat{e}_i$ indica que $e_i$ fue omitida. Esto debido a que la imagen de $e_1\wedge\cdots\wedge e_{i-1}\wedge \hat{e}_i\wedge e_{i+1}\wedge\cdots\wedge e_n$ bajo el diferencial del complejo de Kozsul es

$$ \left(\sum x_ie_i\right)\wedge e_1\wedge\cdots\wedge e_{i-1}\wedge \hat{e}_i\wedge e_{i+1}\wedge\cdots\wedge e_n = (-1)^{i+1}x_ie_1\wedge\cdots\wedge e_n$$

así que el cokernel de $\wedge^{n-1} \rightarrow \wedge^n$ es isomorfo a $R/(x_1,\dots,x_n)$.

En general, como lo sugiere el caso de longitud $2$, la homología del complejo de Koszul guarda relación con las sucesiones regulares. No detectamos siempre si $x_1, \dots, x_n$ es una sucesión regular, pero detectamos algo más interesante: las longitudes de las sucesiones regulares maximales en el ideal $(x_1, \dots, x_n)$. El resultado siguiente también nos dice que todas estas longitudes son la misma.

\begin{theorem}\label{sequence-length-r-theorem}
Sea $M$ un $R$-módulo finitamente generado, y $r \in \mathbb{Z}^*$. Si

$$ H^j(M\otimes K(x_1,\dots,x_n)) = 0 \textrm{ para } j < r $$

mientras que

$$ H^r(M\otimes K(x_1,\dots,x_n)) \neq 0 $$

entonces toda M-sucesión en $I = (x_1,\dots, x_n) \subset R$ tiene longitud $r$.
\end{theorem}

La demostración de este teorema implica definir herramientas que no necesitaremos, pero se puede encontrar a detalle en (Eisenbud referencia).

\begin{corollary}
Si $x_1,\dots,x_n$ es una $M$-sucesión, entonces la sucesión $M\otimes K(x_1,\dots, x_n)$ es exacta excepto en el extremo derecho; es decir, $H^j(M\otimes K(x_1, \dots, x_n)) = 0$ para $j < n$. Más aún, $H^n(M\otimes K(x_1, \dots, x_n)) = M/(x_1, \dots, x_n)$
\end{corollary}
\begin{proof}
La longitud de una $M$-sucesión maximal en $(x_1, \dots, x_n)$  es $l>n$, así que el teorema anterior nos confirma la primera afirmación. Ahora, el producto tensorial como funtor es exacto por la derecha, por lo que preserva cokernels, así que

\begin{align*}
\begin{split}
\textrm{coker}(M\otimes\wedge^{n-1}N\rightarrow M\otimes\wedge^nN) & = M\otimes \textrm{coker}(\wedge^{n-1}N\rightarrow \wedge^nN)\\
& = M \otimes H^n(K(x_1,\dots,x_n))
\end{split}
\end{align*}

Usando los cálculos que hicimos de $H^n(K(x_1,\dots,x_n))$, concluímos que

\begin{align*}
\begin{split}
H^n(M\otimes K(x_1,\dots,x_n)) & = M\otimes R/(x_1,\dots,x_n)\\
& = M/(x_1,\dots,x_n)M
\end{split}
\end{align*}

\end{proof}

Se puede apreciar que el recíproco es falso con el Ejemplo \ref{counterexample-nolocal}, ya que en ningún momento usamos el orden de $x_1,\dots, x_n$, así que podríamos permutar para obtener una sucesión que no es regular; pero veremos más adelante que bajo la hipótesis de anillo local es válido el resultado.

Veamos ahora que si $IM \neq M$, entonces, al menos

$$H^n(M\otimes K(x_1,\dots,x_n)) = M/(x_1,\dots,x_n)M \neq 0$$

mientras, por supuesto,

$$ H^{-1}(M\otimes K(x_1,\dots,x_n)) = 0,$$

así que existe una $r$ para la cual el Teorema \ref{sequence-length-r-theorem} se puede aplicar. Por otro lado, veremos que si $IM = M$, entonces, para toda $j \in Z$, $H^j(M\otimes K(x_1,\dots,x_n)) = 0$.

Si $IM \neq M$, entonces por el Teorema \ref{sequence-length-r-theorem} las longitudes de toda $M$-sucesión maximal en $I$ es la misma. Definimos a la \emph{profundidad} de $I$ en $M$, denotada como $\depth(I,M)$, como la longitud de cualquier $M$-sucesión maximal en $I$. Si $M = R$, simplemente decimos profundidad de $I$. Si $IM = M$, por convención diremos que $\depth(I,M) = \infty$.

La profundidad de $I$ es una especie de medida aritmética del "tamaño" de $I$, mientras que la codimensión de $I$ es una medida geométrica. Veremos que, tal como la codimensión, la profundidad depende únicamente del radical de $I$, y tiene un significado geométrico en el sentido de que, en el caso de los anillos afines, está determinada por las intersecciones de la variedad generada por $I$. Siguiendo esta idea, el teorema \ref{sequence-length-r-theorem} implica un análogo del teorema del ideal principal(que menciona que la altura de cada ideal propio $I$ generado por $n$ elementos es a lo más $n$): Un ideal con $r$ generadores puede tener profundidad a lo más $n$. Más adelante veremos que, en general, $\depth I \leq \codim I$.

Bajo la hipótesis de anillo local podemos dar un resultado más fuerte, y así, tener un criterio para saber si una sucesión en particular es una $M$-sucesión.

\begin{theorem}\label{sequence-criterion-regular}
Sea $R$ anillo local con ideal maximal $\mathfrak{m}$, y sea $M$ un $R$-módulo finitamente generado. Sean $x_1,\dots,x_n\in\mathfrak{m}$. Si para alguna $k \leq n$

$$ H^k(M \otimes K(x_1,\dots,x_n)) = 0, $$

entonces 

$$ (\forall j \leq k)(H^j(M\otimes K(x_1,\dots,x_n))). $$

En particular, si $H^{n-1}(H\otimes K(x_1,\dots,x_n)) = 0$, entonces $x_1,\dots,x_n$ es una $M$-sucesión.
\end{theorem}

La demostración de este teorema implica definir herramientas que no necesitaremos, pero se puede encontrar a detalle en (Eisenbud referencia).

Una consecuencia inmediata de este teorema refuerza al corolario \ref{m-sequence-permutation-corollary}.

\begin{corollary}\label{m-sequence-from-another-criteria}
Sea $R$ anillo local, $M$ un $R$-módulo finitamente generado, y $(x_1,\dots,x_n) \subset R$ un ideal propio que contiene a una $M$-sucesión de longitud $n$, entonces $x_1,\dots,x_n$ es una $M$-sucesión.
\end{corollary}

\begin{proof}
Como $M$ es finitamente generado, el lema de Nakayama muestra que 

$$H^n(M \otimes K(x_1,\dots,x_n)) = M/(x_1,\dots,x_n)M \neq 0.$$

Si $r$ es el entero más pequeño tal que $H^r(M\otimes K(x_1,\dots,x_n)) \neq 0$, entonces, por el teorema \ref{sequence-length-r-theorem}, toda $M$-sucesión maximal en $(x_1,\dots,x_n)$ tiene longitud $r$, y por nuestra hipótesis, $r = n$. Así, $x_1,\dots,x_n$ es una sucesión regular por el teorema \ref{sequence-criterion-regular}.
\end{proof}

Este resultado se puede usar frecuentemente para demostrar que una sucesión dada es regular. Para ejemplificarlo, demostraremos un resultado que nos da cierta intuición geométrica de la profundidad. Antes de eso, necesitaremos un lema.

\begin{lemma}\label{localized-module-not-zero}
Sea $R$ anillo y $M$ un $R$-módulo. Si $M \neq 0$, entonces existen ideales primos y maximales $P$ tales que $M_P \neq 0$.
\end{lemma}

\begin{proof}
Sea $x \in M$, existe algún ideal primo $P$ tal que $(0:x) \subset P$. Al localizar a $M$ en $P$, tenemos que $\frac{x}{1} \neq \frac{0}{1}$, ya que de lo contrario habría $t \in R\setminus P$ tal que $ty = 0$, una contradicción. Cualquier ideal maximal que contenga a $P$ compartirá esta propiedad.
\end{proof}

\begin{corollary}[Naturaleza geométrica de la profundidad]\label{geometry-nature-depth}
Sea $R$ anillo, y $M$ un $R$-módulo finitamente generado.
\begin{enumerate}[a.]
\item Si $x_1,\dots,x_n$ es una $M$-sucesión, y $t_1,\dots,t_n \in \mathbb{Z}^+$, entonces $x_1^{t_1},\dots,x_n^{t_n}$ es una $M$-sucesión.
\item Si $I$ es un ideal de $R$ y $J$ es su radical, tenemos que $\depth(I,M) = \depth(J, M)$.
\end{enumerate}
\end{corollary}

\begin{proof}
\begin{enumerate}[a.]
\item Haremos inducción sobre $r$ para reducir el problema al caso de anillo local. Para $r = 1$ observemos que la potencia de un elemento regular es regular.

Ahora, supongamos que la sucesión $x_1^{t_1},\dots,x_{n-1}^{t_{n-1}}$ es regular. Basta con probar que $x_n$ es regular en $M/(x_1^{t_1},\dots,x_{n-1}^{t_{n-1}})M$, lo que equivale a probar que el automorfismo $\bar{x_n}$ definido por la multiplicación por $x_n$ tiene kernel 0. 

Por la contrapuesta del Lema \ref{localized-module-not-zero} basta probar que para todo ideal primo $P$, el automorfismo inducido en la localización de $M/(x_1^{t_1},\dots,x_{n-1}^{t_{n-1}})M$ en $P$ tiene kernel 0. Eso reduce el problema al caso de anillo local, ya que si $P$ no contiene a algún $x_1,\dots,x_n$, entonces $M/(x_1^{t_1},\dots,x_{n-1}^{t_{n-1}})M = 0$ ó $x_n$ es unidad. Por lo tanto asumiremos que $R$ es un anillo local y $x_1,\dots,x_n$ está contenida en el ideal maximal.

Si $x_1,\dots,x_n$ es una sucesión regular, entonces claramente $x_1,\dots,x_{n-1},x_n^{t_n}$ es regular. Por el corolario \ref{m-sequence-from-another-criteria}, $x_n^{t_n},x_1,\dots,x_{n-1}$ es regular. Iterando este proceso, obtenemos que $x_{n-1}^{t_{n-1}}$, $x_n^{t_n}$, $x_1$, $\dots$, $x_{n-2}$ también es regular. Después de $n$ iteraciones concluímos que $x_1^{t_1},\dots,x_n^{t_n}$ es sucesión regular.

\item Como $I \subset J$, es claro que $\depth(I,M) \leq \depth(J,M)$. La otra desigualdad se da gracias al inciso \emph{a}, ya que si $x_1, \dots, x_n$ es una $M$-sucesión en $J$, entonces existen $t_1,\dots,t_n \in Z^+$ tal que $x_1^{t_1},\dots,x_n^{t_n}$ es $M$-sucesión en $I$.

\end{enumerate}
\end{proof}