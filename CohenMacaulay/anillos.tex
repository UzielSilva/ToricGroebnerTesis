\section{Anillos Cohen-Macaulay}

En la sección anterior hablamos un poco de las propiedades que comparten la profundidad y la codimensión, y mostramos que, para un anillo $R$, y un ideal $I \subset R$ la desigualdad $\depth(I) \leq codim(I)$ siempre se cumple. El siguiente resultado nos ayudará a explotar las propiedades de la igualdad cuando suceda.

\begin{theorem}\label{cohen-macaulay-well-definition}
Sea $R$ anillo tal que para todo $P\subset R$ ideal maximal, $\depth(P) = \codim(P)$. Si $I \subset R$ es un ideal propio, entonces $\depth(I) = \codim(I)$.
\end{theorem}

\begin{proof}
Por la Proposición \ref{depth-leq-codimension} tenemos que $\depth(I) \leq \codim(I)$, probemos la otra desigualdad.

Gracias al Lema \ref{fixed-depth-localization}, podemos localizar en algún ideal maximal $P \supset I$ sin afectar la profundidad de $I$ ni la de $P$, así que asumiremos que $R$ es anillo local con $P$ ideal maximal, e $I \subset P$. 

Analicemos primero el caso en el cual $P$ es ideal primo minimal de $I$ (es decir, no hay un ideal primo $Q$ tal que $I \subset Q \subsetneq P$). Por definición, $\codim(I) = \codim(P)$. Veamos ahora que $\sqrt{I} = P$. Es obvio que $\sqrt{I} \subset P$, así que sólo probaremos la otra contención. Sea $x \not\in \sqrt{I}$ y sea $U = \{1\}\cup\{x^n|n \geq 1\}$(notemos que $U$ es un conjunto multiplicativamente cerrado). Tomemos $Q$ ideal maximal entre los ideales que contienen a $I$ y no tocan a $U$. Por maximalidad, $Q$ es primo, y como $P$ es ideal primo minimal de $I$ y maximal, $P = Q$ y $x \not\in P$. Por el Corolario \ref{geometry-nature-depth}, $\depth(I) = \depth(\sqrt{I}) = \depth(P)$. Así que el teorema se cumple para $I$.

Ahora supongamos que $P$ no es ideal primo minimal de $I$. Podemos asumir por inducción sobre cadenas de ideales que el teorema se cumple para todo ideal estrictamente más grande que $I$. Podemos encontrar por ``evasión de primos''(véase Lema 3.3 de Eisenbud referencia) un elemento $x \in P$ tal que no está contenido en ningún ideal primo minimal de $I$. Por la inducción tenemos que $\depth(I + (x)) = \codim(I + (x)) = \codim(I) + 1$. Pero por el Lema \ref{fixed-depth-localization}, $\depth(I + (x)) \leq \depth(I) + 1$. Así, $\depth(I) \geq \codim(I)$, como se buscaba.
\end{proof}

El Teorema \ref{cohen-macaulay-well-definition} es tan útil, que su hipótesis se ha convertido en una de las definiciones más importantes en el álgebra conmutativa.

\begin{definition}
Un anillo \emph{Cohen-Macaulay} es un anillo tal que para cada ideal maximal $P$ se cumple que $\depth(P) = \codim(P)$.
\end{definition}

La propiedad de ser Cohen-Macaulay es estrictamente local.

\begin{proposition}\label{cohen-macaulay-iff-maximals}
Un anillo $R$ es Cohen-Macaulay sí y sólo si para todo ideal maximal $P$ se cumple que $R_P$ es Cohen-Macaulay, y así, para todo ideal primo $Q$, $R_Q$ es Cohen-Macaulay.
\end{proposition}

\begin{proof}
Si $R$ es Cohen-Macaulay, y $Q$ es un ideal primo, entonces $\codim(Q_Q) = \codim(Q) = \depth(Q,R) \leq \depth(Q_Q) \leq \codim(Q_Q)$ por el Lema \ref{depth-leq-localized} y la Proposición \ref{depth-leq-codimension}, y la desigualdad es una igualdad y $R_Q$ es Cohen-Macaulay.

Si para todo ideal maximal $P$ se tiene que $R_P$ es Cohen-Macaulay, entonces $\depth(P,R) = \depth(P_P, R_P)$ por el Lema \ref{depth-leq-localized}. Como $\codim(P) = \codim(P_P)$, tenemos que $R$ es Cohen-Macaulay.
\end{proof}

La propiedad de ser Cohen-Macaulay también se hereda al anillo de polinomios.

\begin{proposition}
Un anillo $R$ es Cohen-Macaulay sí y sólo si el anillo de polinomios $R[x]$ es Cohen-Macaulay.
\end{proposition}

\begin{proof}
Si $R[x]$ es Cohen-Macaulay, entonces como $x$ es regular, $R[x]/(x) = R$ es Cohen-Macaulay.

Para la otra implicación, gracias la Proposición \ref{cohen-macaulay-iff-maximals} basta con probar que cada localización de $R[x]$ en un ideal maximal es Cohen-Macaulay. Sea $P$ un ideal maximal de $R[x]$, y sea $Q = P\cap R$. Como el complemento de $Q$ en $R$ está contenido en el complemento de $P$ en $R[x]$ tenemos

$$ R[x]_P = R_Q[x]_P $$

Así que asumiremos que $R$ es anillo local con $Q$ ideal maximal. El anillo $R[x]/QR[x] = (R/Q)[x]$ es un dominio de ideales principales, así que, módulo $Q$, el ideal $P$ es generado por un polinomio mónico $f(x)$, es decir, $P=(Q,f(x))$. Si $x_1, \dots, x_n$ es una $R$-sucesión en $Q$, entonces esta es también una $R[x]$-sucesión ya que $R[x]$ es un $R$-módulo libre. Más aún, $f(x)$ es regular módulo cualquier ideal de $R$(puesto que es regular módulo el ideal maximal), así que $x_1,\dots,x_n,f(x)$ es una $R[x]$-sucesión. Así, $\depth(P) \geq 1+\depth(Q)$.

Por otro lado, $\codim(P) = 1 + \codim(Q)$ por el teorema del ideal principal. Como $R$ es Cohen-Macaulay, tenemos que $\codim(Q) = \depth(Q)$; por lo tanto, $\codim(P) \leq \depth(P)$. La desigualdad contraria es inmediata de la Proposición \ref{depth-leq-codimension}.
\end{proof}

Los anillos Cohen-Macaulay generalizan en cierta manera a los \emph{anillos regulares locales}, un anillo local de dimensión $d$ es \emph{regular} si su ideal maximal puede generarse por exactamente $d$ elementos. A esta sucesión de elementos generadores se le llama \emph{sistema regular de parámetros}.

Uno de las propiedades más destacadas de estos anillos es que son dominios enteros(véase Corolario 10.14 del Eisenbud). Partiendo de ello, podemos enunciar el siguiente lema.

\begin{lemma}\label{regular-parameters-sequence}
Si $x_1, \dots, x_d$ es un sistema regular de parámetros de un anillo regular local $R$, entonces, $x_1,\dots, x_d$ es una $R$-sucesión
\end{lemma}

\begin{proof}
Para cada $i$ el anillo $R/(x_1,\dots,x_i)$ es un anillo regular local, y por lo tanto un dominio entero. La imagen de $x_{i+1}$ debe de ser distinta de cero porque de otra manera el ideal maximal de $R$ se generaría con menos elementos.
\end{proof} 

\begin{corollary}
Si $R$ es un anillo regular local, entonces es Cohen-Macaulay.
\end{corollary}

\begin{proof}
Se sigue del Lema \ref{regular-parameters-sequence} y del Teorema \ref{sequence-length-r-theorem}.
\end{proof}