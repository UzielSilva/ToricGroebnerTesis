\section{Profundidad}

Retomando las observaciones de la sección anterior, si tenemos $I$ ideal de un anillo $R$, y $M$ un $R$-módulo, entonces la profundidad de $I$ en $M$ es la longitud de cualquier $M$-sucesión maximal contenida en $I$.

Frecuentemente nos veremos en el caso de localizar un anillo, así que algunas observaciones sobre el comportamiento de la profundidad bajo localizaciones serán de utilidad.

\begin{lemma} \label{depth-leq-localized}
Si $R$ es un anillo, y $P$ es un ideal primo en el soporte de un $R$-módulo finítamente generado $M$. Entonces toda $M$-sucesión en $P$ localiza a una $M_P$ sucesión. De esta manera, para cualquier ideal $I \subset P$ tenemos que $\depth(I,M) \leq \depth(I_P, M_P)$, este último tomado en el anillo $R_P$. En general, la desigualdad puede ser estricta, pero para cualquier ideal $I$, existen ideales maximales $P$ en el soporte de $M$ tales que $\depth(I,M) = \depth(I_P, M_P)$. En particular, si $P$ es un ideal maximal, entonces $\depth(P,M) = \depth(P_P, M_P)$.
\end{lemma}

\begin{proof}
Probemos la primera afirmación, el lema de Nakayama nos garantiza que $I_PM_P \neq M_P$, que es la única parte de la demostración que no es tan evidente. La profundidad puede crecer dado a que la localización $M \rightarrow M_P$ puede anular elementos que anulaban en $M$ a elementos de $I$, por lo tanto, estos elementos de $I$ se volverían regulares.

Para la segunda afirmación, sea $I = (x_1,\dots,x_n)$ y $r = \depth(I, M) \leq n$. Sea $j < r$, por el teorema \ref{sequence-length-r-theorem}, $H^r(M \otimes K(x_1,\dots,x_n)) \neq 0$, $H^j(M \otimes K(x_1,\dots,x_n)) = 0$. Al localizar a $R$ en $P$ inducimos a un nuevo complejo de Koszul, observemos que si $I \not\subset P$, entonces $H^n(M\otimes K(x_1,\dots,x_n))_P \cong M_P/(x_1,\dots,x_n)M_P = 0$, por lo tanto, por el Teorema \ref{sequence-criterion-regular},  $P$ no estaría en el soporte de $H^r(M \otimes K(x_1,\dots,x_n))$. Así, los ideales primos $P$ que contienen a $I$ tales que $\depth(I_P, M_P) = \depth(I, M)$ son exactamente los ideales primos en el soporte de $H^r(M \otimes K(x_1,\dots,x_n))$, el cual es no vacío y contiene ideales maximales gracias al Lema \ref{localized-module-not-zero}. La última afirmación del lema es un caso particular de la segunda afirmación.
\end{proof}

Ya hemos dicho que la profundidad de $I$ es un tipo de medida del tamaño de $I$, tal como la codimensión. En esta sección vamos a explorar la relación entre estas dos medidas. Primero mostraremos que siempre se respeta la desigualdad $\depth(I) \leq \codim(I)$.

Antes de continuar, citaremos un resultado que desempeña en la teoría de la dimensión un papel parecido al del teorema del ideal principal en la teoría de la codimensión. La demostración se puede encontrar en (Eisenbud referencia)

\begin{lemma}\label{fixed-depth-localization}
Si $R$ es un anillo local con $P$ ideal maximal, $M$ es un $R$-módulo finítamente generado, $I$ es u ideal de $R$, y $y \in P$, entonces

$$\depth((I,y),M) \leq \depth(I,M) + 1$$

\end{lemma}

\begin{proposition}\label{depth-leq-codimension}
Sea $R$ anillo con $I$ ideal. Se cumple que la profundidad de $I$ (en $R$) es menor o igual a la codimensión de $I$.
\end{proposition}

\begin{proof}
Sea $x_1,\dots,x_n$ una $M$-sucesión maximal en $I$. Como $x_1$ es regular, no está contenido en ningún ideal primo minimal de $R$, así que $\codim(I/(x_1))$ (como un ideal en $R/(x_1)$) $< codim(I)$. Pero la profundidad de $I/(x_1)$ es $n-1$. Si procedemos inductivamente obtendremos $n-1 \leq \codim(I/(x_1)) < \codim(I)$, dado que $\codim(I/(x_1,\dots,x_n) \neq 0) \leq 1$. 
\end{proof}

